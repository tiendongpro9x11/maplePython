\documentclass[17pt]{extarticle}
\setlength{\parindent}{0pt}
\usepackage[utf8]{vietnam}
\thispagestyle{empty}
\begin{document}
\everymath{\displaystyle}
$\star$ Khảo sát và vẽ đồ thì hàm số: \\
$y=\frac{x^{2} + 2 \; x + 3}{2 \; x + 1}$\\
 Hàm số đã cho có tập xác định là: $R\backslash\left\{\frac{-1}{2}\right\}$\\
 Sự biến thiên của hàm số:\\
	Ta viết hàm số dưới dạng: $y=\frac{1}{2} \; x + \frac{3}{4}+\frac{9}{8 \; x + 4}$\\
 Ta có: \\
$\lim_{x\to -\infty }y=-\infty$ va $\lim_{x\to+\infty }y=+\infty$\\
 Vì $\lim_{x\to(\frac{-1}{2})^{-} }y=-\infty$ và $\lim_{x\to(\frac{-1}{2})^{+} }y=+\infty$ nên đường thẳng $x=\frac{-1}{2}$ là đường thẳng\\
tiệm cận đứng của đồ thị đã cho khi $x\to(\frac{-1}{2})^{-}$ và khi $x\to(\frac{-1}{2})^{+}$\\
 Vì $\lim_{x\to\infty}(\frac{9}{8 \; x + 4})=0$ nên đường thẳng $y=\frac{1}{2} \; x + \frac{3}{4}$ là tiệm cận xiên của\\
 đồ thị hàm số đã cho khi $x\to+\infty$ và $x\to-\infty$\\
$\star$ Bảng biến thiên:\\
 Ta có: y'=$\frac{2 \; x^{2} + 2 \; x - 4}{\left(2 \; x + 1\right)^{2}}$ \\
	y'=0 $\Leftrightarrow$ x=$-2$ hoặc x=$1$ \\
\end{document}