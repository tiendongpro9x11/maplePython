\documentclass[17pt]{extarticle}
\setlength{\parindent}{0pt}
\usepackage[utf8]{vietnam}
\thispagestyle{empty}
\begin{document}
\everymath{\displaystyle}
$\star$ Khảo sát và vẽ đồ thì hàm số: \\
$y=\frac{2 \; x^{2} + 2 \; x + 3}{x + 2}$\\
 Hàm số đã cho có tập xác định là: $R\backslash\left\{-2\right\}$\\
 Sự biến thiên của hàm số:\\
	Ta viết hàm số dưới dạng: $y=2 \; x - 2+\frac{7}{x + 2}$\\
 Ta có: \\
$\lim_{x\to -\infty }y=-\infty$ va $\lim_{x\to+\infty }y=+\infty$\\
 Vì $\lim_{x\to(-2)^{-} }y=-\infty$ và $\lim_{x\to(-2)^{+} }y=+\infty$ nên đường thẳng $x=-2$ là đường thẳng\\
tiệm cận đứng của đồ thị đã cho khi $x\to(-2)^{-}$ và khi $x\to(-2)^{+}$\\
 Vì $\lim_{x\to\infty}(\frac{7}{x + 2})=0$ nên đường thẳng $y=2 \; x - 2$ là tiệm cận xiên của\\
 đồ thị hàm số đã cho khi $x\to+\infty$ và $x\to-\infty$\\
$\star$ Bảng biến thiên:\\
 Ta có: y'=$\frac{2 \; x^{2} + 8 \; x + 1}{\left(x + 2\right)^{2}}$ \\
	y'=0 $\Leftrightarrow$ x=$-2 - \frac{1}{2} \; \sqrt{14}$ hoặc x=$-2 + \frac{1}{2} \; \sqrt{14}$ \\
\end{document}