\documentclass[17pt]{extarticle}
\setlength{\parindent}{0pt}
\usepackage[utf8]{vietnam}
\usepackage[paperheight=6.6in,paperwidth=8.5in,margin=2pt]{geometry}
\begin{document}
\everymath{\displaystyle}
$\star$ Khảo sát và vẽ đồ thì hàm số: \\
$y=\frac{x^{2} - 2 \; x - 3}{x - 2}$\\
 Hàm số đã cho có tập xác định là: $R\backslash\left\{2\right\}$\\
 Sự biến thiên của hàm số:\\
	Ta viết hàm số dưới dạng: $y=x+\frac{-3}{x - 2}$\\
 Ta có: \\
$\lim_{x\to -\infty }y=-\infty$ va $\lim_{x\to+\infty }y=+\infty$\\
 Vì $\lim_{x\to(2)^{-} }y=-\infty$ và $\lim_{x\to(2)^{+} }y=+\infty$ nên đường thẳng $x=2$ là đường thẳng\\
tiệm cận đứng của đồ thị đã cho khi $x\to(2)^{-}$ và khi $x\to(2)^{+}$\\
 Vì $\lim_{x\to\infty}(\frac{-3}{x - 2})=0$ nên đường thẳng $y=x$ là tiệm cận xiên của\\
 đồ thị hàm số đã cho khi $x\to+\infty$ và $x\to-\infty$\\
$\star$ Bảng biến thiên:\\
 Ta có: y'=$\frac{x^{2} - 4 \; x + 7}{\left(x - 2\right)^{2}}$ \\
	y'=$1$ + $\frac{3}{\left(x - 2\right)^{2}}$ > 0 với mọi $x \neq 2$ \\
\end{document}